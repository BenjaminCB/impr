\section{Workshop 3 - Induktion}
I denne workshop kigger vi på \textsc{Mergesort}, som er en sorteringsalgoritme. Algoritmen i pseudokode kan ses i Figur \ref{fig:Mergesort} eller i afsnit 5.4.4 i [Ros]. \textsc{Mergesort} benytter \textsc{Merge}, som kan ses i Figur \ref{fig:Merge} eller i afsnit 5.4.4 i [Ros]. Bemærk at den her er beskrevet på en lidt anden måde, men at metoden og tanken bag er den samme. Inden I begynder på selve opgaverne kan I med fordel læse of forstå afsnit 5.4.4. Som algoritmen er beskrevet herunder vil man starte med at kalde \textsc{Mergesort}($L=(a_0,a_1,\ldots,a_{n-1}) $, $0$, $n-1$).

\begin{figure}[h!]
		\begin{algorithmic}
		\Procedure{Mergesort}{$L=(a_0,a_1,\ldots,a_{n-1}) $, $l$, $r$}
		\If{$l<r$} 
		\State $m=\floor{\frac{r+l}{2}}$
		\State $\textsc{Mergesort}(L, l, m)$
		\State $\textsc{Mergesort}(L, m+1, r)$
		\State $L= \textsc{Merge}(L, l, m,r)$\EndIf
		\Return $L$
		\EndProcedure
	\end{algorithmic}
	\caption{Mergesort}\label{fig:Mergesort}
\end{figure}

\begin{figure}[h!]
		\begin{algorithmic}
		\Procedure{Merge}{$L,l,m,r$}
		\State $L_1=L[l,l+1,\ldots,m]$
		\State $L_2=L[m+1,m+2,\ldots,r]$
		\State $i=0$
		\State $j=0$
			\While{$i<m-l+1$ and $j<r-m$}
				\If{$L_1[i]\leq L_2[j]$} \State $L[l+i+j]=L_1[i]$\State $i=i+1$
				\Else ~$L[l+i+j]=L_2[j]$\State $j=j+1$
				\EndIf
			\EndWhile
			\If{$i=m-l+1$} \For{$k=j\ldots r-m-1$}
							\State$L[l+i+k]=L_2[k]$
						\EndFor
			\Else \For{$k=i\ldots m-l$}
							\State$L[l+j+k]=L_1[k]$
						\EndFor
			\EndIf
			\Return $L$
		\EndProcedure
	\end{algorithmic}
	\caption{Merge}\label{fig:Merge}
\end{figure}
