\subsection{Delopgave 1}
\subsubsection{}
\textbf{Task:} Implementer \textsc{Merge} og \textsc{Mergesort}. 

\bigskip
\noindent
\textbf{Solution:} Taget fra https://www.geeksforgeeks.org/c-program-for-merge-sort/ fordi den fra moodle på ingen måde gad at virke uanset hvor meget jeg rettede i det. Den er alligevel baseret på næsten en direkt kopi af denne, så det går nok.
\begin{lstlisting}
#include<stdlib.h> 
#include<stdio.h> 
    
void merge(int arr[], int l, int m, int r) 
{ 
    int i, j, k; 
    int n1 = m - l + 1; 
    int n2 =  r - m; 
  
    /* create temp arrays */
    int L[n1], R[n2]; 
  
    /* Copy data to temp arrays L[] and R[] */
    for (i = 0; i < n1; i++) 
        L[i] = arr[l + i]; 
    for (j = 0; j < n2; j++) 
        R[j] = arr[m + 1+ j]; 
  
    /* Merge the temp arrays back into arr[l..r]*/
    i = 0; // Initial index of first subarray 
    j = 0; // Initial index of second subarray 
    k = l; // Initial index of merged subarray 
    while (i < n1 && j < n2) 
    { 
        if (L[i] <= R[j]) 
        { 
            arr[k] = L[i]; 
            i++; 
        } 
        else
        { 
            arr[k] = R[j]; 
            j++; 
        } 
        k++; 
    } 
  
    /* Copy the remaining elements of L[], if there 
       are any */
    while (i < n1) 
    { 
        arr[k] = L[i]; 
        i++; 
        k++; 
    } 
  
    /* Copy the remaining elements of R[], if there 
       are any */
    while (j < n2) 
    { 
        arr[k] = R[j]; 
        j++; 
        k++; 
    } 
} 

void mergeSort(int arr[], int l, int r) 
{ 
    if (l < r) 
    { 
        // Same as (l+r)/2, but avoids overflow for 
        // large l and h 
        int m = l+(r-l)/2; 
  
        // Sort first and second halves 
        mergeSort(arr, l, m); 
        mergeSort(arr, m+1, r); 
  
        merge(arr, l, m, r); 
    } 
} 

void printArray(int A[], int size) 
{ 
    int i; 
    for (i=0; i < size; i++) 
        printf("%d ", A[i]); 
    printf("\n"); 
} 

int main() 
{ 
    int arr[] = { 5, 3, 8, 1, 6, 10, 7, 2, 4, 9 }; 
    int arr_size = sizeof(arr)/sizeof(arr[0]); 
  
    printf("Given array is \n"); 
    printArray(arr, arr_size); 
  
    mergeSort(arr, 0, arr_size - 1); 
  
    printf("\nSorted array is \n"); 
    printArray(arr, arr_size); 
    return 0; 
} 
\end{lstlisting}
\subsubsection{}
\textbf{Task:} Brug \textsc{Mergesort} til at sortere listen $L=(5,3,8,1,6,10,7,2,4,9)$
\bigskip
\noindent
\textbf{Solution:}

\begin{lstlisting}
    benjamin@DESKTOP-CNN41EU:~/c_opgaver$ ./mergekopi
    Given array is
    5 3 8 1 6 10 7 2 4 9
    Sorted array is
    1 2 3 4 5 6 7 8 9 10
\end{lstlisting}


